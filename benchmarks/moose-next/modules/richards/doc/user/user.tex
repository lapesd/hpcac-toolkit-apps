\documentclass[]{scrreprt}
\usepackage{amsmath,amsfonts,graphicx}
\usepackage{multirow}
\usepackage{pslatex}
\usepackage{tabularx}
\usepackage{comment}
\usepackage{xspace}
\usepackage{array}

\usepackage{hyperref}

\usepackage{caption}
\DeclareCaptionFont{white}{\color{white}}
\DeclareCaptionFormat{listing}{\colorbox{gray}{\parbox{\textwidth}{#1#2#3}}}

\graphicspath{
{figures/}
}

\newcommand{\uo}{\mbox{UO\textsubscript{2}}\xspace}

\setcounter{secnumdepth}{3}


\begin{document}


\title{Richards Examples}
\author{Andy Wilkins \\
CSIRO}
\maketitle

\tableofcontents

%%%
\chapter{Introduction}
%%%

The Richards' equation describes slow fluid flow through a porous
medium.  This document outlines input-file examples for the Richards
MOOSE code, drawing upon the test suite, and provides guidelines for
creating models that run smoothly.  There are two other accompanying
documents: (1) The theoretical and numerical foundations of the code,
which also describes the notation used throughout this document; (2)
The test suite, which describes the benchmark tests used to validate
the code.

Each example is located in the {\em test} directory, which has path
\begin{verbatim}
<install_dir>/moose/modules/richards/tests
\end{verbatim}


\chapter{User Object examples}

In the test suite there are a number of tests that allow easy
exploration of the functional form of the User Objects.  These are:
\begin{enumerate}
\item {\tt relperm.i} outputs the relative permeability as a function
  of effective saturation.  The user may specify which particular form
  of relative permeability to output, and the appropriate parameters
  of interest.
\item {\tt density.i} outputs the fluid density as a function of fluid
  pressure.  The user may specify which particular form of density to
  output, and the appropriate parameters of interest.
\item {\tt seff1.i} outputs the effective saturation for a
  single-phase situation as a function of porepressure (if
  porepressure is positive then the effective saturation is unity).
  The user may specify which particular form of effective saturation
  to output, and the appropriate parameters of interest.
\item {\tt seff2.i} outputs the effective saturation for a 2-phase
  situation as a function of differences in porepressure (the
  difference is, say, $P_{\mathrm{gas}} - P_{\mathrm{water}}$, which
  is positive).  The user may specify which particular form of
  effective saturation to output, and the appropriate parameters of
  interest.
\end{enumerate}
In all cases, an exodus file is output that can be used to generate
plots of the user objects.  Also a CSV file is output that contains
the user object evaluated at a specific point of interest.


\chapter{Convergence and convergence criteria}

As a general rule, the formulation of multiphase flow implemented in
MOOSE is quite suitable to solve many different problems.  However,
there are some situations where the implementation is not optimal,
such as the tracking of fronts.  MOOSE may converge in these
instances, but may take unacceptably short time steps, or it may not
even converge at all.  In these cases it is probably best to either
modify the problem to something more suitable, or use another code.

For example, in simulations with sharp fronts, the user should ask
whether it is truly necessary to accurately track the sharp front, or
whether similar results could be taken by smoothing the fronts by
using slightly different initial conditions, and/or by modifying the
effective-saturation relationships, by making the van Genuchten
parameter $\alpha$ smaller, for instance.  If tracking sharp fronts is
vital to the problem then a code specifically designed to solve such
problems will do it better than MOOSE, or perhaps MOOSE could be used
with an ALE front-tracking algorithm.

Below I list some general pointers that may help with problems that
MOOSE is struggling with.
\begin{itemize}
\item External fluxes that turn off or on too quickly are bad.
  Simialrly, those that have large discontinuities in their
  derivatives can cause convergence problems.  Try to define
  ``smooth'' versions of these as inputs.  Discontinuities like these
  often manifest themselves in non-convergence of the nonlinear
  iterations.
\item Effective saturation curves that are too ``flat'' are not good.
  For example, the van Genuchten parameter $m=0.6$ almost always gives
  better convergence than $m=0.9$.
\item Effective saturation curves that are too ``low'' are not good.
  For example, the van Genuchten parameter $\alpha=10^{-6}$\,Pa$^{-1}$
  almost always gives better convergence than
  $\alpha=10^{-3}$\,Pa$^{-1}$.  Compare, for instance, the tests {\tt
    buckley\_leverett/bl21.i} and {\tt buckley\_leverett/bl22.i}.
\item Any discontinuities in the effective saturation, or its
  derivative, are bad.  I suggest using van Genuchten parameter
  $\alpha\geq 0.5$ for problems with both saturated and unsaturated
  zones if the van Genuchten relative permeability relation is used.
\item Highly nonlinear relative permeability curves make convergence
  difficult in some cases.  For instance, a ``power'' relative
  permeability curve with $n=20$ is much worse numerically than with
  $n=2$.  See if you can reduce the nonlinearity in your curve.
\item Any discontinuities in the relative permeability, or its derivative,
  are bad.  For most curves coded into MOOSE this is not an issue, but
  I recommend the {\tt RichardsRelPermVG1} curve over {\tt
    RichardsRelPermVG}, since the former is smooth around
  $S_{\mathrm{eff}}=1$.  See {\tt recharge\_discharge/rd01.i} in the
  tests directory for an example of this.   I also suggest using the
  ``power'' covers over the ``VG'' curves.
\item In multiphase problems if one phase completely disappears, MOOSE
  may not converge, as discussed more fully in
  Chapter~\ref{chap.sing.jac}.  To avoid this:
\begin{enumerate}
\item The fully-upwind kernel and boundary fluxes and dirac sources
  can be used.  If the immobile saturation of the phase is nonzero,
  then it probably won't disappear, as the fully-upwind approach will
  not, in theory, allow fluid to exit from a node if the relative
  permeability is zero.  However, numerical imprecision can lead to
  phase disappearance.
\item A nonzero residual saturation can be used.  This means that for
  $dt\rightarrow 0$ the Jacobian matrix will be nonsingular.  (If the
  residual saturations are zero then the Jacobian is singular for
  $dt\rightarrow 0$.)  Then in most cases the problematic node will
  fill with a little amount of the phase in the next time step.
\item A ``shifted'' van Genuchten capillary suction curve may be used
  in difficult multiphase problems.
\end{enumerate}
\item When porosity is a function of time, small time steps may be
  necessary, especially in multiphase situations.  MOOSE will
  converge, but may need quite a few nonlinear iterations (maybe about
  20), and the PETSc tolerances should be set quite tight, otherwise
  MOOSE might ``converge'' to a crazy solution that causes
  difficulties in subsequent time steps.
\item Check whether the linear or nonlinear solvers are causing the
  problem.  If it is the latter, and you have addressed the potential
  problems above (most particularly discontinuities, and phase
  disappearence), then probably your problem is just highly
  nonlinear.  However, you can often address problems with the
  {\em linear} solver not converging quickly by choosing different
  preconditioners and ksp methods.  I typically use one of:
\begin{enumerate}
\item {\tt -ksp\_type=gmres -pc\_type=bjacobi}
\item {\tt -ksp\_type=gmres -pc\_type=asm -sub\_pc\_type=lu -sub\_pc\_factor\_shift\_type=NONZERO}
\item {\tt -ksp\_type=gmres -pc\_type=asm -sub\_pc\_type=lu -sub\_pc\_factor\_shift\_type=NONZERO -pc\_asm\_overlap=2 -ksp\_diagonal\_scale -ksp\_diagonal\_scale\_fix -ksp\_gmres\_modifiedgramschmidt}
\end{enumerate}
\end{itemize}

Choosing reasonable convergence criteria is very important.  The
Theory Manual contains a section that explains the {\em minimum}
residual that a user can expect to obtain in a model.  However, this
minimum is usually much smaller than what is important from a
practical point of view.  If a tiny residual is chosen, MOOSE will
spend most of its time changing pressure values by tiny amounts as it
attempts to converge to the tiny residual, and this is a waste of
compute time.   This problem may be amplified if adaptive time-stepping is
used since MOOSE doesn't realise that most of the compute time is
spent ``doing nothing'', so keeps the timestep very small.  So, here
are some guidelines for choosing an appropriate tolerance on the
residual.
\begin{enumerate}
\item Determine an appopriate tolerance on what you mean by
  ``steadystate''.  For instance, in a single-phase simulation with
  reasonably large constant fluid bulk modulus, and gravity acting in
  the $-z$ direction, the steadystate solution is $P = -\rho_{0}gz$
  (up to a constant).  In the case of water, this reads $P=-10000z$.
  Instead of this, suppose you would be happy to say the model is at
  steadystate if $P = -(\rho_{0} g + \epsilon)z$.  For instance, for water,
  $\epsilon=1$\,Pa.m$^{-1}$ might be suitable in your problem.  Then recall
  that the residual is just
\begin{equation}
R = \left|\int
\nabla_{i}\left(\frac{\kappa_{ij}\kappa_{\mathrm{rel}}\rho}{\mu}(\nabla_{j}P
+ \rho g_{j}) \right) \right|
\label{eqn.res.int}
\end{equation}
Evaluate this for your ``steadystate'' solution.  For instance, in the
case of water just quoted, $R = V|\kappa|\rho_{0}/\mu\epsilon =
V|\kappa|\times 10^{6}\epsilon$, where: $V$ is the volume of the
finite-element mesh, and I have inserted standard values for
$\rho_{0}$ and $\mu$.
\item In the previous step, an appropriate tolernace on the residual
  was given as $V|\kappa|\rho_{0}\epsilon/\mu$.  However, this is often too
  large because of the factor of $V$.  The previous step assumed that
  the solution was incorrect by a factor, $\epsilon$, which is constant
  over the entire mesh.   More commonly, there is a small region of
  the mesh where most of the interesting dynamics occurs, and the
  remainder of the mesh exists just to provide reasonable boundary
  conditions for this ``interesting'' region.  The residual in the
  ``boring'' region can be thought of as virtually zero, while the
  residual in the ``interesting'' region is
  $V_{\mathrm{interesting}}|\kappa|\rho_{0}\epsilon/\mu$.  This is smaller
  than the residual in the previous step, so provides a tighter
  tolerance for MOOSE to strive towards.
\item In the previous steps, I've implicitly assumed $\kappa$ is
  constant, $\rho$ is virtually constant at $\rho_{0}$, only a
  single-phase situation, etc.  In many cases these assumptions are
  not valid, so the integral of Eqn~(\ref{eqn.res.int}) cannot be done
  as trivially as in the previous steps.  In these cases, I simply
  suggest to build a model with initial conditions like $P =
  -(\rho_{0} g + \epsilon)z$, and just see what the initial residual is.
  That will give you an idea of how big a reasonable residual
  tolerance should be.
\end{enumerate}


\chapter{Two-phase, almost saturated}

If a two-phase model has regions that are fully saturated with the
``1'' phase (typically this is water), then the residual for the ``2''
phase is zero.  This means the ``2''-phase pressure will not change in
those regions, potentially violating $P_{1}\leq P_{2}$.  If the ``2''
phase subsequently infiltrates to these regions, an initially crazy
$P_{2}$ might affect the results.  This sometimes also holds for
almost-saturated situations, depending on the exact simulation.

In these cases, probably the best way of avoiding problems is to
implement the constraint $P_{1}\leq P_{2}$ using a {\tt
  RichardsMultiphaseProblem} Problem object.  In fact, I almost always
use this as standard in my 2-phase simulations, just in case one phase
disappears.  This gets around the
problem of chosing the $a$ in the penalty term described in the next
paragraph.  See Section~\ref{sec.bound.pp} for more comments.

Another way is to add a penalty term to the residual to
ensure that $P_{1}\leq P_{2}$.  An example can be found in the tests
directory {\tt pressure\_pulse/pp22.i}.  The choice of the $a$
parameter is sometimes difficult: too big and the penalty term
dominates the Darcy flow; too small and the penalty term does
nothing.  In both cases, convergence is poor as the penalty term
switches on and off during the Newton-Raphson procedure.  The
documentation for {\tt RichardsPPenalty} describes how to set $a$ (run
MOOSE with a {\tt -\,-dump} flag).

The penalty term should {\em not} be used unless absolutely necessary
as it will lead to poorer convergence characteristics.  In many cases
it is not necessary.

As mentioned above, in these situations it is often advantageous to
use the fully-upwind kernels, boundary fluxes and dirac sources.  If
the immobile saturation of the phase is nonzero then it probably won't
disappear entirely.  If a phase disappears entirely then the Jacobian
may be singular for $dt\rightarrow 0$, and this sometimes leads to
nonconvergence.  This can be avoided by using a nonzero residual
saturation, or a shifted van Genuchten relationship.  Please see
Chapter~\ref{chap.sing.jac} for more discussions and tips.






\chapter{Bounding porepressure}
\label{sec.bound.pp}

Occasionally it might be useful to bound porepressure.  The test {\tt
  buckley\_leverett/bl22.i} has ``bounds'' that do this.  Note that:
\begin{itemize}
\item The convergence is likely to be much slower when using bounds
\item The {\tt -snes\_type} must be set to {\tt vinewtonssls} (see the
  {\tt [Preconditioning]} block of the aforementioned test).
\item The command line must contain the argument {\tt --use-petsc-dm}.
\end{itemize}

\noindent The ``bounds'' just described uses PETSc to enforce $P\geq a$, for
some fixed $a$.  In two-phase situations, one often wants to
enforce $P_{1}\geq P_{2}$.  To do this a {\tt
  RichardsMultiphaseProblem} object can be used.  See for example {\tt
  gravity\_head\_2/gh\_bounded\_17.i}, which may be compared to {\tt
  gravity\_head\_2/gh\_lumped\_17.i} to see how easy it is to use
this in an input file.  Note that:
\begin{itemize}
\item The convergence is likely to be much slower when using a {\tt
  RichardsMultiphaseProblem} object.
\item The use of this object will likely avoid crazy behaviour of a
  gas phase's porepressure when the phase is completely or almost
  completely absent in a region.
\item Care must be taken: in the example {\tt gh\_bounded\_17.i}, I
  had to use zero residual saturations, otherwise the simulation keeps
  trying to reduce {\tt pgas} below {\tt pwater} to conserve gas
  mass.  If you see your nonlinear residual not decreasing, you might
  be using {\tt RichardsMultiphaseProblem} to enforce a constraint
  which does not make physical sense (eg, it causes nonconservation of mass).
\end{itemize}




\chapter{Singular Jacobians}
\label{chap.sing.jac}

In multiphase simulations, there are subtleties associated with one of
the phases reducing to its residual saturation, and the resulting
problems encountered can completely dominate the convergence
characteristics of simulations.  They can be understood by considering
a simulation containing a single node, with no Darcy flux.  The
residual is just
\begin{equation}
R = \frac{d}{dt}\left(
\begin{array}{l}
\phi \rho_{g}S_{g} \\
\phi \rho_{w}S_{w}
\end{array}
\right) \ .
\end{equation}
Here I've labelled the phases ``g'' and ``w'', for gas and water.
With the variables being $P_{g}$ and $P_{w}$, the
Jacobian is
\begin{equation}
J = \frac{\phi}{dt} \left(
\begin{array}{cc}
\rho_{g}'S_{g} + \rho_{g}\frac{\partial S_{g}}{\partial P_{g}} &
\rho_{g}\frac{\partial S_{g}}{\partial P_{w}} \\
\rho_{w}\frac{\partial S_{w}}{\partial P_{g}} &
\rho_{w}'S_{w} + \rho_{w}\frac{\partial S_{w}}{\partial P_{w}}
\end{array}
\right)
= \frac{\phi}{dt} \left(
\begin{array}{cc}
\rho_{g}'S_{g} + \rho_{g}S' &
-\rho_{g}S' \\
-\rho_{w}S' &
\rho_{w}'S_{w} + \rho_{w}S'
\end{array}
\right)
\end{equation}
Now imagine that $S_{g} = S_{g}^{\mathrm{res}}$ (that gas has reduced
to the residual value).  Using the standard van-Genuchten expression
for $S'$ gives
\begin{equation}
S' = 0
\end{equation}
and so
\begin{equation}
J = \frac{\phi}{dt} \left(
\begin{array}{cc}
\rho_{g}'S_{g}^{\mathrm{res}}  &
0 \\
0 &
\rho_{w}'(1 - S_{g}^{\mathrm{res}})
\end{array}
\right)
\end{equation}
Note that if $S_{g}^{\mathrm{res}}=0$, then this Jacobian is singular!

This singularity will manifest itself in various ways:
\begin{itemize}
\item When one phase disappears PETSc will find it difficult or
  impossible to invert the Jacobian, leading to a large number of
  linear iterations
\item The nonlinear solver will find it difficult or impossible to
  converge.
\end{itemize}

\noindent The arguments above strongly point to ensuring that simulations are
always run with
\begin{equation}
S^{\mathrm{res}}>0.
\end{equation}
for all phases.  Then the Jacobian is non-singular if the timestep
sizes are not too big, so that the time-derivative terms dominate the
Richards flux terms.  {\em However, this can also lead to
  difficulties.}  Imagine that the `g' phase is at its residual
value.  Then the difficulties encountered are:
\begin{enumerate}
\item If the `w' phase wants to increase its pressure (because of
  Richards' flux, say), then the `g' phase pressure {\em must
    increase} at least as fast as the `w' phase to maintain $P_{g}\geq
  P_{w}$.  However, if there is no `g' phase entering the node
  (because of immobile saturation, say), mass conservation is thereby
  violated.  This was mentioned in Chapter~\ref{sec.bound.pp} in
  relation to the test {\tt gh\_bounded\_17.i}.  The nonlinear
  iterations in MOOSE will typically not converge in this case.
\item If the porosity decreases, and the `g' phase is more
  compressible than the `w' phase ($\rho_{g}'/\rho_{g} >
  \rho_{w}'/\rho_{w}$) the `g' phase saturation will attempt to reduce
  below its residual value.  In some ways this is unfair to Richards'
  flow, as the whole concept of residual saturation was never designed
  to handle dynamic porosity, but nevertheless this problem will
  manifest itself in nonconvergence of the nonlinear solver.
\end{enumerate}
The existence of these two cases (which are not too unusual in real
simulations) have made it necessary to create the ``shifted''
van-Genuchten capillary curve for two-phase situations.

The ``shifted'' van-Genuchten capillary curve is documented in the
Theory and Test manuals.  It has the important property that
\begin{equation}
S' > 0 \ ,
\end{equation}
for all finite porepressures.  Therefore, the determinant of the
Jacobian is always positive
\begin{equation}
\det J = \rho_{g}'\rho_{w}'S(1-S) + \rho_{g}'\rho_{w}SS' +
\rho_{g}\rho_{w}'(1-S)S' > 0 \ .
\end{equation}
since $0\leq S\leq 1$, and $\rho'>0$ physically, so the Jacobian is
non-singular in the full Richards case for $\mathrm{d}t \rightarrow 0$.  (Here
$S = S_{g} = 1-S_{w}$.)   Finally, using $S^{\mathrm{res}}=0$ gets
around both of the problems mentioned in the previous paragraph.








\end{document}

